\documentclass{article}
% Packages
\usepackage[section]{placeins}
\usepackage[top=2cm, bottom=2cm, left=2cm, right=2cm]{geometry}
\usepackage[round]{natbib}

% Title
\begin{document}
\title{Decomposing population dynamics}
\author{In alphabetic order: Koen van Benthem, Timothée Bonnet, Marjolein Bruijning}
\maketitle

% Begin main text
\section*{Introduction} 
Different processes shape population dynamics.\\
Recently, rapid evolution has received a lot of attention, as a possibly important factor influencing population dynamics on a short time-scale.\\
Rapid evolution has been demonstrated in many terrestrial and aquatic plants/animals.\\
Ecology and evolution thus cannot longer be considered as separate, independent processes. \\
However, it has been proven problematic to disentangle both processes in natural or experimental populations.\\
While this is important in predicting population responses towards changing environments. \\
In this paper, we discuss frameworks that exist to decompose population dynamics into its underlying causes. \\
We discuss assumptions that have to be met, and differences and similarities between these frameworks.\\
We will give examples of how these frameworks have been applied to natural systems. \\
Three major frameworks we discuss are: quantitative genetics (mainly focussing on the animal model), the recently proposed GPE-equation (Ellner 2011) and the recently proposed age-structured Price-equation (Coulson 2008).\\
We then apply the frameworks to a simulated dataset and compare the outcomes.\\
We finish by an evaluation of the frameworks, and discuss whether there is a way to unify the different frameworks.

\section*{Overview of existing frameworks}
\subsection*{The animal model}
Has been widely used in many natural systems.\\
It is based on the partition of variance components.\\
It enables estimation of additive genetic variation, environmental variation, variation through maternal effects, heritability. \\
The animal model is individually based, and is a mixed effects model, in which these variance components can be estimated.\\
Genetic relationships between \emph{all} individuals are taken into account.\\
Important assumptions: no interaction effect between environment and genotypes, no correlation between environment and genotypes, environmental variance independent of genotype, no interaction between loci, normally distributed variables. \\
However: although phenotypic variance can be decomposed into sources of variation, it has not been applied on a population level.

\subsection*{Age-structured Price equation}
blabla
\subsection*{GPE-equation}
blabla

\section*{Material and Methods}
\subsection*{Data simulation}
We simulated a dataset of a hypothetical species. \\
Following processes were included: 

\subsection*{Applying the frameworks}
blabla

\section*{Results}
\subsection*{Selection differentials}
\subsection*{Additive genetic variance}
\subsection*{The importance of ecology vs evolution}

\section*{Discussion}
blabla
\end{document}